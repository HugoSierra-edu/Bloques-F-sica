\documentclass{standalone}
\usepackage{tikz}
\usepackage{amsmath} % Para usar \vec
\usetikzlibrary{arrows.meta, shapes.geometric, calc}

\begin{document}

% --- Definición de colores ---
\colorlet{DarkRed}{red!60!black}
\colorlet{DarkBlue}{blue!80!black}

\begin{tikzpicture}[
    % --- Estilos ---
    axis/.style={-Latex, thick, color=black},
    field_vector/.style={-{Stealth[length=3mm, width=2.5mm]}, color=DarkRed!80!black, line width=1.2pt},
    total_vector/.style={-{Stealth[length=4mm, width=3.5mm]}, color=DarkRed, line width=1.8pt},
    pos_vector/.style={-Stealth, color=DarkBlue, thick, dashed},
    star_shape/.style={star, star points=7, star point ratio=0.7, draw=red!80!black, fill=red!40, minimum size=10pt},
    point/.style={circle, draw=DarkBlue, fill=blue!30, minimum size=6pt, inner sep=0pt}
]

% --- Coordenadas ---
\def\d{1.2}
\coordinate (A) at (-3*\d, 0);
\coordinate (B) at (3*\d, 0);
\coordinate (P) at (0, 4*\d);
\coordinate (O) at (0,0);

% --- Ejes y Objetos ---
\draw[axis] (-4.5*\d, 0) -- (4.5*\d, 0) node[below left] {$x$};
\draw[axis] (0, -1.5*\d) -- (0, 5.2*\d) node[below left] {$y$};
\node[star_shape, label=below:$A$] at (A) {};
\node[star_shape, label=below:$B$] at (B) {};
\node[point, label=above right:$P$] at (P) {};
\node at (O) [cross, draw, minimum size=3pt, thick] {};

% --- Etiquetas de distancia ---
\draw[|<->|, thin] ($(A)+(0,-0.2)$) -- ($(O)+(0,-0.2)$) node[midway, below] {$3d$};
\draw[|<->|, thin] ($(O)+(0,-0.2)$) -- ($(B)+(0,-0.2)$) node[midway, below] {$3d$};
\draw[|<->|, thin] ($(O)+(0.2,0)$) -- ($(P)+(0.2,0)$) node[midway, right] {$4d$};

% --- Vectores de Posición ---
\draw[pos_vector] (A) -- (P) node[midway, above left, sloped, font=\small, color=black] {$\vec{r}_{AP}$};
\draw[pos_vector] (B) -- (P) node[midway, above right, sloped, font=\small, color=black] {$\vec{r}_{BP}$};

% --- Vectores de Campo y su Suma ---
% 1. Definimos los factores de escala y proporción de masa
\def\scaleFactor{0.3}
\def\massRatio{0.6} % m_B / m_A

% 2. Calculamos los vectores usando la sintaxis robusta
\coordinate (vec_gA) at ($ \scaleFactor*(A) - \scaleFactor*(P) $);
\coordinate (vec_gB) at ($ \massRatio*\scaleFactor*(B) - \massRatio*\scaleFactor*(P) $);

% 3. Dibujamos los vectores de campo individuales desde P
\draw[field_vector] (P) -- ++(vec_gA) node[anchor=east, xshift=-2pt] {$\vec{g}_A$};
\draw[field_vector] (P) -- ++(vec_gB) node[anchor=west, xshift=2pt] {$\vec{g}_B$};

% 4. Calculamos el punto final del vector suma y lo dibujamos
\coordinate (P_total) at ($(P) + (vec_gA) + (vec_gB)$);
\draw[total_vector] (P) -- (P_total) node[below right, yshift=-2pt] {$\vec{g}_T$};

% 5. (Opcional) Dibujamos las líneas de ayuda del paralelogramo
\draw[gray, dotted, thin] ($(P)+(vec_gA)$) -- (P_total);
\draw[gray, dotted, thin] ($(P)+(vec_gB)$) -- (P_total);

\end{tikzpicture}

\end{document}