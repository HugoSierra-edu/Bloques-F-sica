%%%%%%%%%%%%%%%%%%%%%%%%%%%%%%%%%%%%%%%%%%%%%%%%%%%%%%%%%%%%%%%
%
% Welcome to Overleaf --- just edit your LaTeX on the left,
% and we'll compile it for you on the right. If you open the
% 'Share' menu, you can invite other users to edit at the same
% time. See www.overleaf.com/learn for more info. Enjoy!
%
%%%%%%%%%%%%%%%%%%%%%%%%%%%%%%%%%%%%%%%%%%%%%%%%%%%%%%%%%%%%%%%
% Author: Izaak Neutelings (March 2020)
\documentclass[border=3pt,tikz]{standalone}
\usepackage{amsmath} % for \dfrac
\usepackage{physics}
\usepackage{tikz,pgfplots}
\usepackage{tikz-3dplot}
\usepackage[outline]{contour} % glow around text
\usetikzlibrary{angles,quotes} % for pic (angle labels)
\usetikzlibrary{arrows,arrows.meta}
\usetikzlibrary{calc}
\usetikzlibrary{decorations.markings}
\tikzset{>=latex} % for LaTeX arrow head
\usepackage{xcolor}
\colorlet{veccol}{green!45!black}
\colorlet{Bcol}{violet!90}
\colorlet{Bcol1}{violet!80!blue!90}
\colorlet{Bcol2}{violet!80!red!90}
\colorlet{BFcol}{red!60!black}
\colorlet{veccol}{green!45!black}
\colorlet{Icol}{blue!70!black}
\tikzstyle{BField}=[->,thick,Bcol]
\tikzstyle{current}=[->,Icol,thick]
\tikzstyle{force}=[->,thick,BFcol]
\tikzstyle{vector}=[->,thick,veccol]
\tikzstyle{velocity}=[->,very thick,vcol]
\tikzstyle{charge+}=[very thin,draw=black,top color=red!50,bottom color=red!90!black,shading angle=20,circle,inner sep=0.5]
\tikzstyle{charge-}=[very thin,draw=black,top color=blue!50,bottom color=blue!80,shading angle=20,circle,inner sep=0.5]
\tikzstyle{metal}=[top color=black!15,bottom color=black!25,middle color=black!20,shading angle=10]
\tikzstyle{darkmetal}=[top color=black!40,bottom color=black!70,middle color=black!30,shading angle=10]
\tikzset{
  BFieldLine/.style={thick,Bcol,decoration={markings,mark=at position #1 with {\arrow{latex}}},
                                 postaction={decorate}},
  BFieldLine/.default=0.5,
  pics/Bin/.style={
    code={
      \def\R{0.12}
      \draw[pic actions,line width=0.6,#1,fill=white] % ,thick
        (0,0) circle (\R) (-135:.75*\R) -- (45:.75*\R) (-45:.75*\R) -- (135:.75*\R);
  }},
  pics/Bout/.style={
    code={
      \def\R{0.12}
      \draw[pic actions,line width=0.6,#1,fill=white] (0,0) circle (\R);
      \fill[pic actions,#1] (0,0) circle (0.3*\R);
  }},
  pics/Bin/.default=Bcol,
  pics/Bout/.default=Bcol,
}
\tikzstyle{measure}=[fill=white,midway,outer sep=2]
\contourlength{1.4pt}

% RING SHADING
\makeatletter
\pgfdeclareradialshading[tikz@ball]{ring}{\pgfpoint{0cm}{0cm}}%
{rgb(0cm)=(1,1,1);
rgb(0.719cm)=(1,1,1);
color(0.72cm)=(tikz@ball);
rgb(0.9cm)=(1,1,1)}
\tikzoption{ring color}{\pgfutil@colorlet{tikz@ball}{#1}\def\tikz@shading{ring}\tikz@addmode{\tikz@mode@shadetrue}}
\makeatother


\begin{document}


% TWO WIRES ATTRACTION
\def\NB{5}
\def\L{5.5}
\def\W{0.1}
\def\d{1.2}
\begin{tikzpicture}
  
  % MAGNETIC FIELD
  \foreach \i [evaluate={\x=(\i-0.5)*\L/\NB;}] in {1,...,\NB}{
    \pic at (\x, 0.25*\d) {Bout={Bcol1}};
    \pic at (\x, 0.75*\d) {Bin={Bcol2}};
    \pic at (\x,-0.25*\d-\W) {Bin={Bcol1}};
    \pic at (\x, 1.25*\d+\W) {Bout={Bcol2}};
  }
  \node[Bcol1] at (\L,0.22*\d) {$\vb{B}_1$};
  \node[Bcol2] at (\L,0.75*\d) {$\vb{B}_2$};
  
  % VECTORS
  \draw[metal] (0,0) rectangle ++(\L,-\W);
  \draw[metal] (0,\d) rectangle ++(\L,\W);
  \draw[current] (0.92*\L,-\W/2) --++ (0.18*\L,0) node[right=-1] {$I_1$};
  \draw[current] (0.92*\L,\d+\W/2) --++ (0.18*\L,0) node[right=-1] {$I_2$};
  \draw[force] (0.55*\L,-\W/2) --++ (0,0.44*\d) node[below right,scale=0.9] {$F_{21}$};
  \draw[force] (0.55*\L,\d+\W/2) --++ (0,-0.44*\d) node[above right,scale=0.9] {$F_{12}$};
  
\end{tikzpicture}


% TWO WIRES REPULSION
\begin{tikzpicture}
  
  % MAGNETIC FIELD
  \foreach \i [evaluate={\x=(\i-0.5)*\L/\NB;}] in {1,...,\NB}{
    \pic at (\x, 0.25*\d) {Bout={Bcol1}};
    \pic at (\x, 0.75*\d) {Bout={Bcol2}};
    \pic at (\x,-0.25*\d-\W) {Bin={Bcol1}};
    \pic at (\x, 1.25*\d+\W) {Bin={Bcol2}};
  }
  \node[Bcol1] at (\L,0.22*\d) {$\vb{B}_1$};
  \node[Bcol2] at (\L,0.75*\d) {$\vb{B}_2$};
  
  % VECTORS
  \draw[metal] (0,0) rectangle ++(\L,-\W);
  \draw[metal] (0,\d) rectangle ++(\L,\W);
  \draw[current] (0.92*\L,-\W/2) --++ (0.18*\L,0) node[right=-1] {$I_1$};
  \draw[current] (0.08*\L,\d+\W/2) --++ (-0.18*\L,0) node[left=-1] {$I_2$};
  \draw[force] (0.55*\L,-\W/2) --++ (0,-0.44*\d) node[above=2,right,scale=0.9] {$F_{21}$};
  \draw[force] (0.55*\L,\d+\W/2) --++ (0,0.44*\d) node[below=2,right,scale=0.9] {$F_{12}$};
  
\end{tikzpicture}



\end{document}