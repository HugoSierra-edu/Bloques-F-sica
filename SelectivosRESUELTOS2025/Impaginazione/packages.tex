% ======================================================================
% ARCHIVO DE PAQUETES CENTRALIZADO Y UNIFICADO
% Impaginazione/packages.tex
% ======================================================================

% --- GENERALES Y DE DOCUMENTO ---
\usepackage[utf8]{inputenc}
\usepackage[T1]{fontenc}
\usepackage[inner=4cm, outer=3cm]{geometry}
\usepackage{enumitem}

% --- CONFIGURACIÓN DE IDIOMA Y SILABEO (CORREGIDA Y SIMPLIFICADA) ---
% La opción 'main=spanish' es la forma moderna de asegurar que el idioma
% principal y sus reglas de silabeo (guiones) se carguen correctamente.
\usepackage[main=spanish]{babel}
\usepackage{csquotes}
\usepackage{graphicx}
\usepackage{listings}
\usepackage[svgnames]{xcolor}
\usepackage{comment}
\usepackage{setspace}
\usepackage{appendix}
\usepackage{lipsum}

% --- MATEMÁTICAS ---
\usepackage{amsmath}
\usepackage{amssymb}
\usepackage{physics}
\usepackage{braket}

% --- PORTADA, CABECERAS Y PIES DE PÁGINA ---
\usepackage{frontespizio}
\usepackage{tikz}
\usepackage{fancyhdr}
\pagestyle{fancy}
\setlength{\headheight}{24.1638pt}
\fancyhf{}
\fancyhead[LE]{\leftmark}
\fancyhead[RO]{\rightmark}
\fancyfoot[LE,RO]{\thepage}

% --- PAQUETES AÑADIDOS PARA EL FORMATO DE LAS SOLUCIONES ---
\usepackage[many]{tcolorbox}
\usepackage{float}
\usepackage{eso-pic}
\usepackage{hyperref}
\hypersetup{ colorlinks=true, linkcolor=black, urlcolor=blue }

% --- BIBLIOGRAFÍA, ÍNDICES Y GLOSARIOS ---
\usepackage[style=numeric]{biblatex}
\addbibresource{biblio.bib}

\newenvironment{abstracto}% <--- He cambiado el nombre a "abstracto" para no entrar en conflicto
    {\cleardoublepage%
        \thispagestyle{empty}%
        \null \vfill\begin{center}%
            \bfseries Resumen \end{center}}%
        {\vfill\null}

\usepackage{imakeidx}
\makeindex
\usepackage{nomencl}
\makenomenclature
\usepackage{glossaries}
\makeglossaries

% --- CONFIGURACIONES PERSONALIZADAS PARA LAS CAJAS ---
\newtcolorbox{cajaenunciado}{
    fonttitle=\bfseries, colframe=black, colback=white, boxrule=0.5pt, arc=0mm, title=Enunciado del Problema
}
\newtcolorbox{cajaresultado}{
    fonttitle=\bfseries, colframe=black, colback=white, boxrule=0.5pt, arc=0mm, title=Resultado Final
}
\newtcolorbox{cajaconclusion}{
    fonttitle=\bfseries, colframe=black, colback=white, boxrule=0.5pt, arc=0mm, title=Conclusión
}
% --- CONFIGURACIÓN PARA EL FORMATO DE TÍTULOS DE CAPÍTULO ---
\usepackage{titlesec}

\titleformat{\chapter}[display]
  {\normalfont\Large\bfseries\centering} % Formato general del bloque
  {\chaptertitlename\ \thechapter}        % El texto "Capítulo X"
  {0pt}                                  % Separación horizontal (no necesaria)
  {\vspace{1em}}                         % Espacio vertical para "bajar" el número

\titlespacing*{\chapter}
  {0pt}   % Sangría izquierda
  {20pt}  % Espacio VERTICAL ANTES del título (reducido para subirlo)
  {40pt}  % Espacio VERTICAL DESPUÉS del título

% --- PAQUETES PARA DIBUJO VECTORIAL (AÑADIDO) ---
\usepackage{tikz}
\usetikzlibrary{arrows.meta, patterns, decorations.pathmorphing, calc}

