% !TEX root = ../main.tex
\chapter{Examen Junio 2000 - Convocatoria Ordinaria}
\label{chap:2000_jun_ord}

% ======================================================================
\section{Bloque I: Interacción Gravitatoria}
\label{sec:grav_2000_jun_ord}
% ======================================================================

\subsection{Cuestión 1 - OPCIÓN A}
\label{subsec:1A_2000_jun_ord}

\begin{cajaenunciado}
Para los planetas del sistema solar, según la tercera ley de Kepler, la relación $R^{3}/T^{2}$ es constante y vale $3,35\times10^{18}\,\text{m}^{3}/\text{s}^{2}$, siendo R el radio de sus órbitas y T el periodo de rotación. Suponiendo que las órbitas son circulares, calcular la masa del Sol.
\textbf{Dato:} $G=6,67\times10^{-11}$ S.I.
\end{cajaenunciado}
\hrule

\subsubsection*{1. Tratamiento de datos y lectura}
Los datos proporcionados en el enunciado son:
\begin{itemize}
    \item \textbf{Constante de Kepler para el Sistema Solar ($K_S$):} $K_S = \frac{R^3}{T^2} = 3,35 \cdot 10^{18} \, \text{m}^3/\text{s}^2$.
    \item \textbf{Constante de Gravitación Universal (G):} $G = 6,67 \cdot 10^{-11} \, \text{N}\cdot\text{m}^2/\text{kg}^2$.
    \item \textbf{Incógnita:} La masa del Sol ($M_S$).
\end{itemize}

\subsubsection*{2. Representación Gráfica}
Se representa un planeta de masa $m$ orbitando alrededor del Sol (masa $M_S$). La fuerza gravitatoria es la responsable del movimiento circular, actuando como fuerza centrípeta.
\begin{figure}[H]
    \centering
    \fbox{\parbox{0.6\textwidth}{\centering \textbf{Órbita planetaria} \vspace{0.5cm} \textit{Prompt para la imagen:} "Un esquema simple del sistema solar con el Sol en el centro. Un planeta se muestra en una órbita circular de radio R. Dibujar un vector para la fuerza de atracción gravitatoria, $F_g$, ejercida por el Sol sobre el planeta, apuntando hacia el Sol. Añadir una etiqueta que indique que esta fuerza actúa como fuerza centrípeta, $F_c$." \vspace{0.5cm} 
    }}
    \caption{Esquema de la fuerza gravitatoria como fuerza centrípeta.}
\end{figure}

\subsubsection*{3. Leyes y Fundamentos Físicos}
Para resolver el problema, se deben aplicar dos principios fundamentales:
\begin{itemize}
    \item \textbf{Ley de Gravitación Universal de Newton:} La fuerza de atracción entre dos masas $M$ y $m$ separadas una distancia $R$ es $F_g = G \frac{M m}{R^2}$.
    \item \textbf{Dinámica del Movimiento Circular Uniforme (MCU):} Para que un cuerpo de masa $m$ describa una órbita circular de radio $R$ con un periodo $T$, debe estar sometido a una fuerza centrípeta $F_c = m a_c = m \omega^2 R$. La velocidad angular $\omega$ se relaciona con el periodo mediante $\omega = \frac{2\pi}{T}$.
\end{itemize}
En el caso de una órbita planetaria, la fuerza gravitatoria es la fuerza centrípeta ($F_g = F_c$).

\subsubsection*{4. Tratamiento Simbólico de las Ecuaciones}
Igualamos las expresiones de la fuerza gravitatoria y la fuerza centrípeta para un planeta de masa $m$ que orbita alrededor del Sol ($M_S$):
\begin{gather}
    F_g = F_c \nonumber \\
    G \frac{M_S m}{R^2} = m \omega^2 R = m \left(\frac{2\pi}{T}\right)^2 R
\end{gather}
La masa del planeta, $m$, se cancela en ambos lados de la ecuación. Reorganizamos la expresión para agrupar los términos conocidos:
\begin{gather}
    G \frac{M_S}{R^2} = \frac{4\pi^2}{T^2} R \nonumber \\
    G M_S = 4\pi^2 \frac{R^3}{T^2}
\end{gather}
Despejamos la masa del Sol, $M_S$:
\begin{gather}
    M_S = \frac{4\pi^2}{G} \left(\frac{R^3}{T^2}\right)
\end{gather}
Esta expresión relaciona la masa del Sol con la constante de gravitación $G$ y la constante de la tercera ley de Kepler, que es el dato proporcionado.

\subsubsection*{5. Sustitución Numérica y Resultado}
Sustituimos los valores numéricos de las constantes en la ecuación final:
\begin{gather}
    M_S = \frac{4\pi^2}{6,67 \cdot 10^{-11} \, \text{N}\cdot\text{m}^2/\text{kg}^2} \left(3,35 \cdot 10^{18} \, \text{m}^3/\text{s}^2\right) \approx 1,98 \cdot 10^{30} \, \text{kg}
\end{gather}
\begin{cajaresultado}
    La masa del Sol calculada es $\boldsymbol{M_S \approx 1,98 \cdot 10^{30} \, \textbf{kg}}$.
\end{cajaresultado}

\subsubsection*{6. Conclusión}
\begin{cajaconclusion}
    Al igualar la fuerza de atracción gravitatoria con la fuerza centrípeta necesaria para mantener una órbita circular, se deduce una expresión para la masa del cuerpo central. Utilizando la constante de Kepler proporcionada, se determina que la masa del Sol es de aproximadamente $1,98 \times 10^{30}$ kilogramos, un valor consistente con las mediciones astronómicas actuales.
\end{cajaconclusion}

\newpage

\subsection{Cuestión 1 - OPCIÓN B}
\label{subsec:1B_2000_jun_ord}

\begin{cajaenunciado}
Enumera y comenta las interacciones que conozcas.
\end{cajaenunciado}
\hrule

\subsubsection*{1. Tratamiento de datos y lectura}
Se trata de una cuestión teórica que no requiere datos numéricos. La tarea consiste en describir las cuatro interacciones fundamentales de la naturaleza.

\subsubsection*{2. Representación Gráfica}
Se proponen esquemas conceptuales para cada una de las cuatro interacciones.
\begin{figure}[H]
    \centering
    \fbox{\parbox{0.45\textwidth}{\centering \textbf{Gravitatoria y Electromagnética} \vspace{0.5cm} \textit{Prompt para la imagen:} "Dos paneles. Arriba: La Tierra deformando el tejido del espaciotiempo y la Luna siguiendo esa curvatura, representando la gravedad. Abajo: Dos cargas eléctricas, una positiva y otra negativa, con líneas de campo entre ellas y un fotón (gamma) viajando entre ambas, representando la interacción electromagnética." \vspace{0.5cm} % \includegraphics[width=0.9\linewidth]{interacciones_macro.png}
    }}
    \hfill
    \fbox{\parbox{0.45\textwidth}{\centering \textbf{Nuclear Fuerte y Débil} \vspace{0.5cm} \textit{Prompt para la imagen:} "Dos paneles. Arriba: Un núcleo atómico con protones y neutrones unidos. Un gluón (espiral) se intercambia entre dos quarks dentro de un protón, representando la interacción fuerte. Abajo: Un neutrón convirtiéndose en un protón, emitiendo un electrón y un antineutrino, con un bosón W mediando la transformación, representando la interacción débil." \vspace{0.5cm} % \includegraphics[width=0.9\linewidth]{interacciones_micro.png}
    }}
    \caption{Representación conceptual de las cuatro interacciones fundamentales.}
\end{figure}

\subsubsection*{3. Leyes y Fundamentos Físicos}
En la física moderna, se considera que toda la materia y sus interacciones se pueden describir en términos de cuatro fuerzas o interacciones fundamentales. A continuación, se describen en orden de intensidad decreciente.

\paragraph{1. Interacción Nuclear Fuerte:}
Es la más intensa de todas. Su principal característica es que mantiene unidos los quarks para formar protones y neutrones, y a su vez, mantiene unidos a estos protones y neutrones para formar el núcleo atómico, venciendo la repulsión eléctrica entre los protones.
\begin{itemize}
    \item \textbf{Partículas mediadoras:} Gluones.
    \item \textbf{Alcance:} Muy corto, del orden del tamaño de un núcleo atómico ($\approx 10^{-15}$ m). Fuera de esta distancia, su efecto es nulo.
    \item \textbf{Actúa sobre:} Quarks y, por extensión, hadrones (protones, neutrones).
\end{itemize}

\paragraph{2. Interacción Electromagnética:}
Es responsable de los fenómenos eléctricos y magnéticos. A nivel atómico, es la fuerza que liga los electrones a los núcleos para formar átomos, y la que permite la formación de moléculas a través de enlaces químicos. A escala macroscópica, es la causa de las fuerzas de contacto, el rozamiento, y la mayoría de las fuerzas que experimentamos en la vida diaria.
\begin{itemize}
    \item \textbf{Partícula mediadora:} Fotón.
    \item \textbf{Alcance:} Infinito. Su intensidad disminuye con el cuadrado de la distancia.
    \item \textbf{Actúa sobre:} Partículas con carga eléctrica.
\end{itemize}

\paragraph{3. Interacción Nuclear Débil:}
Es la responsable de ciertos procesos de desintegración nuclear, como la desintegración beta, en la que un neutrón se convierte en un protón (o viceversa). Juega un papel crucial en las reacciones nucleares que tienen lugar en las estrellas, como el Sol.
\begin{itemize}
    \item \textbf{Partículas mediadoras:} Bosones $W^+$, $W^-$ y $Z^0$.
    \item \textbf{Alcance:} Extremadamente corto, incluso menor que el de la interacción fuerte ($\approx 10^{-18}$ m).
    \item \textbf{Actúa sobre:} Quarks y leptones (electrones, neutrinos).
\end{itemize}

\paragraph{4. Interacción Gravitatoria:}
Es la más débil de todas, pero a grandes escalas (planetas, galaxias) es la fuerza dominante debido a su naturaleza exclusivamente atractiva y su alcance infinito. Describe la atracción entre cuerpos con masa. Es la responsable de la estructura a gran escala del universo, las órbitas planetarias y la caída de los cuerpos.
\begin{itemize}
    \item \textbf{Partícula mediadora (teórica):} Gravitón (aún no detectado experimentalmente).
    \item \textbf{Alcance:} Infinito. Su intensidad disminuye con el cuadrado de la distancia.
    \item \textbf{Actúa sobre:} Todas las partículas con masa y/o energía.
\end{itemize}

\subsubsection*{6. Conclusión}
\begin{cajaconclusion}
Las cuatro interacciones fundamentales (fuerte, electromagnética, débil y gravitatoria) gobiernan todos los fenómenos del universo. Difieren enormemente en intensidad, alcance y las partículas sobre las que actúan. Mientras que las interacciones nucleares dominan a escala subatómica, el electromagnetismo rige la química y la materia ordinaria, y la gravedad estructura el cosmos a gran escala.
\end{cajaconclusion}

\newpage

% ======================================================================
\section{Bloque II: Ondas}
\label{sec:ondas_2000_jun_ord}
% ======================================================================

\subsection{Problema 1 - OPCIÓN A}
\label{subsec:2A_2000_jun_ord}

\begin{cajaenunciado}
Dos fuentes sonoras emiten ondas armónicas planas no amortiguadas de igual amplitud y frecuencia. Si la frecuencia es de 2000 Hz y la velocidad de propagación es de $340\,\text{m/s}$, determinar la diferencia de fase en un punto del medio de propagación situado a 8 m de una fuente y a 25 m de la otra fuente sonora. Razonar si se producirá interferencia constructiva o destructiva en dicho punto.
\end{cajaenunciado}
\hrule

\subsubsection*{1. Tratamiento de datos y lectura}
Se extraen y organizan los datos, convirtiéndolos al SI si fuera necesario.
\begin{itemize}
    \item \textbf{Frecuencia de las ondas ($f$):} $f = 2000 \, \text{Hz}$.
    \item \textbf{Velocidad de propagación ($v$):} $v = 340 \, \text{m/s}$.
    \item \textbf{Distancia a la primera fuente ($r_1$):} $r_1 = 8 \, \text{m}$.
    \item \textbf{Distancia a la segunda fuente ($r_2$):} $r_2 = 25 \, \text{m}$.
    \item \textbf{Incógnitas:} Diferencia de fase ($\Delta\phi$) en el punto de interés y tipo de interferencia.
\end{itemize}

\subsubsection*{2. Representación Gráfica}
Se dibuja un esquema que muestra las dos fuentes y el punto donde se estudia la interferencia.
\begin{figure}[H]
    \centering
    \fbox{\parbox{0.7\textwidth}{\centering \textbf{Interferencia de Ondas} \vspace{0.5cm} \textit{Prompt para la imagen:} "Un diagrama con dos puntos etiquetados como 'Fuente 1' (S1) y 'Fuente 2' (S2). Un tercer punto, P, está situado en el plano. Dibujar líneas rectas desde S1 hasta P (etiquetada como $r_1=8$ m) y desde S2 hasta P (etiquetada como $r_2=25$ m). Mostrar frentes de onda circulares emanando de S1 y S2 para ilustrar la propagación." \vspace{0.5cm} % \includegraphics[width=0.7\linewidth]{interferencia_fuentes.png}
    }}
    \caption{Esquema del fenómeno de interferencia entre dos fuentes sonoras.}
\end{figure}

\subsubsection*{3. Leyes y Fundamentos Físicos}
El fenómeno de la interferencia ocurre cuando dos o más ondas coinciden en un punto del espacio. La perturbación resultante es la suma de las perturbaciones individuales (Principio de Superposición).
\begin{itemize}
    \item La \textbf{diferencia de fase ($\Delta\phi$)} entre dos ondas que provienen de fuentes en fase depende de la diferencia de caminos recorridos ($\Delta r = |r_2 - r_1|$).
    \item La relación es $\Delta\phi = k \cdot \Delta r$, donde $k = \frac{2\pi}{\lambda}$ es el número de onda.
    \item La \textbf{longitud de onda ($\lambda$)} se relaciona con la velocidad y la frecuencia mediante $\lambda = v/f$.
    \item \textbf{Interferencia constructiva:} Ocurre cuando $\Delta\phi = 2n\pi$ rad (con $n=0, 1, 2, \dots$). Esto equivale a $\Delta r = n\lambda$. La amplitud resultante es máxima.
    \item \textbf{Interferencia destructiva:} Ocurre cuando $\Delta\phi = (2n+1)\pi$ rad (con $n=0, 1, 2, \dots$). Esto equivale a $\Delta r = (n+\frac{1}{2})\lambda$. La amplitud resultante es mínima (o nula si las amplitudes iniciales son iguales).
\end{itemize}

\subsubsection*{4. Tratamiento Simbólico de las Ecuaciones}
\paragraph{a) Cálculo de la longitud de onda ($\lambda$)}
A partir de la relación fundamental de las ondas:
\begin{gather}
    \lambda = \frac{v}{f}
\end{gather}
\paragraph{b) Cálculo de la diferencia de fase ($\Delta\phi$)}
Primero se calcula la diferencia de caminos $\Delta r$:
\begin{gather}
    \Delta r = |r_2 - r_1|
\end{gather}
Luego, se utiliza la relación entre la diferencia de fase y la diferencia de caminos:
\begin{gather}
    \Delta\phi = k \cdot \Delta r = \frac{2\pi}{\lambda} \Delta r = \frac{2\pi f}{v} \Delta r
\end{gather}

\subsubsection*{5. Sustitución Numérica y Resultado}
\paragraph{a) Valor de la longitud de onda}
\begin{gather}
    \lambda = \frac{340 \, \text{m/s}}{2000 \, \text{Hz}} = 0,17 \, \text{m}
\end{gather}
\paragraph{b) Valor de la diferencia de caminos y fase}
\begin{gather}
    \Delta r = |25 \, \text{m} - 8 \, \text{m}| = 17 \, \text{m}
\end{gather}
Ahora calculamos la diferencia de fase:
\begin{gather}
    \Delta\phi = \frac{2\pi}{0,17 \, \text{m}} (17 \, \text{m}) = 200\pi \, \text{rad}
\end{gather}
\begin{cajaresultado}
    La diferencia de fase en el punto especificado es $\boldsymbol{\Delta\phi = 200\pi \, \textbf{rad}}$.
\end{cajaresultado}

\paragraph{c) Tipo de interferencia}
La diferencia de fase obtenida, $200\pi$, es de la forma $2n\pi$ con $n=100$. Esta es la condición para una interferencia constructiva.
Alternativamente, podemos comparar la diferencia de caminos con la longitud de onda:
$$ \frac{\Delta r}{\lambda} = \frac{17 \, \text{m}}{0,17 \, \text{m}} = 100 $$
Como la diferencia de caminos es un múltiplo entero ($n=100$) de la longitud de onda ($\Delta r = 100\lambda$), la interferencia es constructiva.

\begin{cajaresultado}
    En dicho punto se producirá una \textbf{interferencia constructiva}.
\end{cajaresultado}

\subsubsection*{6. Conclusión}
\begin{cajaconclusion}
La diferencia de caminos entre las dos fuentes es de 17 metros, lo que corresponde exactamente a 100 longitudes de onda. Esto resulta en una diferencia de fase de $200\pi$ radianes. Al ser un múltiplo par de $\pi$, las ondas llegan en fase al punto de observación, produciendo una interferencia constructiva y, por lo tanto, una máxima intensidad sonora.
\end{cajaconclusion}

\newpage

\subsection{Problema 1 - OPCIÓN B}
\label{subsec:2B_2000_jun_ord}

\begin{cajaenunciado}
Una onda armónica plana que se propaga en el sentido positivo del eje OX, tiene un periodo de 0,2 s. En un instante dado, la diferencia de fase entre dos puntos separados una distancia de 60 cm es igual a $\pi$ radianes. Se pide determinar:
\begin{enumerate}
    \item Longitud de onda y velocidad de propagación de la onda.
    \item Diferencia de fase entre dos estados de perturbación de un mismo punto que tienen lugar en dos instantes separados por un intervalo de tiempo de 2 s.
\end{enumerate}
\end{cajaenunciado}
\hrule

\subsubsection*{1. Tratamiento de datos y lectura}
\begin{itemize}
    \item \textbf{Periodo de la onda ($T$):} $T = 0,2 \, \text{s}$.
    \item \textbf{Separación espacial ($\Delta x$):} $\Delta x = 60 \, \text{cm} = 0,6 \, \text{m}$.
    \item \textbf{Diferencia de fase espacial ($\Delta\phi_x$):} $\Delta\phi_x = \pi \, \text{rad}$ para el $\Delta x$ dado.
    \item \textbf{Intervalo de tiempo ($\Delta t$):} $\Delta t = 2 \, \text{s}$.
    \item \textbf{Incógnitas:}
    \begin{enumerate}
        \item Longitud de onda ($\lambda$) y velocidad de propagación ($v$).
        \item Diferencia de fase temporal ($\Delta\phi_t$) para el $\Delta t$ dado.
    \end{enumerate}
\end{itemize}

\subsubsection*{2. Representación Gráfica}
\begin{figure}[H]
    \centering
    \fbox{\parbox{0.45\textwidth}{\centering \textbf{Apartado 1: Fase Espacial} \vspace{0.5cm} \textit{Prompt para la imagen:} "Gráfico de una onda sinusoidal ($y$ vs $x$) en un instante $t_0$. Marcar dos puntos en el eje x, $x_1$ y $x_2$, separados por $\Delta x = 60$ cm. Los puntos correspondientes en la onda, $(x_1, y_1)$ y $(x_2, y_2)$, deben estar en oposición de fase (por ejemplo, un máximo y un mínimo). Indicar que la diferencia de fase entre ellos es $\pi$ rad." \vspace{0.5cm} % \includegraphics[width=0.9\linewidth]{fase_espacial.png}
    }}
    \hfill
    \fbox{\parbox{0.45\textwidth}{\centering \textbf{Apartado 2: Fase Temporal} \vspace{0.5cm} \textit{Prompt para la imagen:} "Gráfico de una oscilación sinusoidal ($y$ vs $t$) en una posición fija $x_0$. Marcar dos puntos en el eje t, $t_1$ y $t_2$, separados por $\Delta t = 2$ s. Indicar la diferencia de fase $\Delta\phi_t$ entre los estados de oscilación en esos dos instantes." \vspace{0.5cm} % \includegraphics[width=0.9\linewidth]{fase_temporal.png}
    }}
    \caption{Representación de las diferencias de fase espacial y temporal.}
\end{figure}

\subsubsection*{3. Leyes y Fundamentos Físicos}
La fase de una onda armónica que se propaga en el sentido +OX viene dada por $\phi(x,t) = kx - \omega t + \phi_0$.
\begin{itemize}
    \item \textbf{Apartado 1:} La diferencia de fase entre dos puntos $x_1$ y $x_2$ en un mismo instante $t$ es $\Delta\phi_x = k(x_2-x_1) = k \Delta x$. El número de onda es $k = 2\pi/\lambda$.
    \item La velocidad de propagación se relaciona con la longitud de onda y el periodo mediante $v = \lambda/T$.
    \item \textbf{Apartado 2:} La diferencia de fase para un mismo punto $x$ entre dos instantes $t_1$ y $t_2$ es $\Delta\phi_t = \omega(t_2-t_1) = \omega \Delta t$. La frecuencia angular es $\omega = 2\pi/T$.
\end{itemize}

\subsubsection*{4. Tratamiento Simbólico de las Ecuaciones}
\paragraph{1. Longitud de onda y velocidad de propagación}
De la relación de la diferencia de fase espacial:
\begin{gather}
    \Delta\phi_x = \frac{2\pi}{\lambda} \Delta x \implies \lambda = \frac{2\pi \Delta x}{\Delta\phi_x}
\end{gather}
Una vez obtenida $\lambda$, la velocidad de propagación es:
\begin{gather}
    v = \frac{\lambda}{T}
\end{gather}

\paragraph{2. Diferencia de fase temporal}
Usando la definición de frecuencia angular:
\begin{gather}
    \Delta\phi_t = \omega \Delta t = \frac{2\pi}{T} \Delta t
\end{gather}

\subsubsection*{5. Sustitución Numérica y Resultado}
\paragraph{1. Cálculo de $\lambda$ y $v$}
Sustituimos los datos en la ecuación para la longitud de onda:
\begin{gather}
    \lambda = \frac{2\pi \cdot (0,6 \, \text{m})}{\pi \, \text{rad}} = 1,2 \, \text{m}
\end{gather}
Ahora calculamos la velocidad de propagación:
\begin{gather}
    v = \frac{1,2 \, \text{m}}{0,2 \, \text{s}} = 6 \, \text{m/s}
\end{gather}
\begin{cajaresultado}
    La longitud de onda es $\boldsymbol{\lambda = 1,2 \, \textbf{m}}$ y la velocidad de propagación es $\boldsymbol{v = 6 \, \textbf{m/s}}$.
\end{cajaresultado}

\paragraph{2. Cálculo de $\Delta\phi_t$}
Sustituimos los datos en la ecuación para la diferencia de fase temporal:
\begin{gather}
    \Delta\phi_t = \frac{2\pi}{0,2 \, \text{s}} (2 \, \text{s}) = 20\pi \, \text{rad}
\end{gather}
\begin{cajaresultado}
    La diferencia de fase temporal es $\boldsymbol{\Delta\phi_t = 20\pi \, \textbf{rad}}$.
\end{cajaresultado}

\subsubsection*{6. Conclusión}
\begin{cajaconclusion}
A partir de la relación entre la separación espacial y la diferencia de fase, se ha determinado una longitud de onda de 1,2 m y una velocidad de 6 m/s. Para la evolución temporal de un punto, un intervalo de 2 segundos, que equivale a 10 periodos completos ($2s / 0,2s = 10$), corresponde a una diferencia de fase de $20\pi$ radianes. Esto significa que el punto se encuentra exactamente en el mismo estado de oscilación en ambos instantes.
\end{cajaconclusion}

\newpage

% ======================================================================
\section{Bloque III: Óptica}
\label{sec:optica_2000_jun_ord}
% ======================================================================

\subsection{Cuestión 1 - OPCIÓN A}
\label{subsec:3A_2000_jun_ord}

\begin{cajaenunciado}
Dada una lente delgada convergente, obtener de forma gráfica la imagen de un objeto situado entre el foco y la lente. Indicar las características de dicha imagen.
\end{cajaenunciado}
\hrule

\subsubsection*{1. Tratamiento de datos y lectura}
La cuestión es de naturaleza gráfica y teórica. Los elementos a considerar son:
\begin{itemize}
    \item \textbf{Elemento óptico:} Lente delgada convergente.
    \item \textbf{Posición del objeto ($s$):} Entre el foco objeto ($F$) y el centro óptico ($O$). Es decir, $|s| < f$.
    \item \textbf{Tarea:} Construir el diagrama de rayos y describir la imagen resultante.
\end{itemize}

\subsubsection*{2. Representación Gráfica}
El diagrama de rayos es la solución principal a la cuestión.
\begin{figure}[H]
    \centering
    \fbox{\parbox{0.9\textwidth}{\centering \textbf{Formación de imagen en lente convergente (lupa)} \vspace{0.5cm} \textit{Prompt para la imagen:} "Diagrama de trazado de rayos para una lente delgada convergente. Dibuja el eje óptico horizontal. Coloca la lente vertical en el centro, con su centro óptico O. Marca el foco objeto F a la izquierda y el foco imagen F' a la derecha, a igual distancia de O. Dibuja un objeto vertical (una flecha apuntando hacia arriba) situado entre F y O. Traza los siguientes tres rayos principales desde la punta del objeto: 1) Un rayo paralelo al eje óptico que, tras refractarse en la lente, pasa por el foco imagen F'. 2) Un rayo que pasa por el centro óptico O y no se desvía. 3) Un rayo que se dirige hacia la lente como si proviniera del foco objeto F y, tras refractarse, emerge paralelo al eje óptico. Los tres rayos refractados divergen. Prolonga estos rayos refractados hacia atrás (a la izquierda de la lente) con líneas discontinuas. El punto donde estas prolongaciones se cruzan es la punta de la imagen. Dibuja la imagen como una flecha vertical discontinua. Etiqueta claramente el objeto (y), la imagen (y'), los focos (F, F') y la lente." \vspace{0.5cm} % \includegraphics[width=0.9\linewidth]{lente_convergente_lupa.png}
    }}
    \caption{Trazado de rayos para un objeto situado entre el foco y una lente convergente.}
\end{figure}

\subsubsection*{3. Leyes y Fundamentos Físicos}
La construcción de la imagen se basa en el modelo de rayos y la óptica geométrica para lentes delgadas. Se utilizan los "rayos principales", cuyo comportamiento al atravesar la lente es conocido:
\begin{enumerate}
    \item \textbf{Rayo paralelo al eje:} Todo rayo que incide paralelo al eje óptico, se refracta pasando por el foco imagen ($F'$).
    \item \textbf{Rayo que pasa por el centro óptico:} Todo rayo que pasa por el centro óptico ($O$) no sufre desviación.
    \item \textbf{Rayo que pasa por el foco objeto:} Todo rayo que pasa por el foco objeto ($F$) antes de incidir en la lente, se refracta emergiendo paralelo al eje óptico.
\end{enumerate}
La imagen se forma en el punto donde los rayos refractados (o sus prolongaciones) se cruzan. Si se cruzan los propios rayos, la imagen es real. Si se cruzan sus prolongaciones, la imagen es virtual.

\subsubsection*{4. Características de la Imagen}
Observando el diagrama de rayos construido en el apartado gráfico, se pueden determinar las características de la imagen formada:
\begin{itemize}
    \item \textbf{Naturaleza:} La imagen se forma por la prolongación de los rayos refractados, no por los rayos mismos. Por lo tanto, la imagen es \textbf{virtual}. No se podría proyectar sobre una pantalla.
    \item \textbf{Orientación:} La imagen está orientada en el mismo sentido que el objeto (ambas flechas apuntan hacia arriba). Por lo tanto, la imagen es \textbf{derecha} o directa.
    \item \textbf{Tamaño:} La imagen es visiblemente más grande que el objeto. Por lo tanto, la imagen es \textbf{de mayor tamaño} o aumentada.
\end{itemize}
Esta configuración corresponde al funcionamiento de una \textbf{lupa}.

\subsubsection*{6. Conclusión}
\begin{cajaconclusion}
Cuando un objeto se sitúa entre el foco y el centro óptico de una lente convergente, el trazado de rayos demuestra que se forma una imagen \textbf{virtual, derecha y de mayor tamaño} que el objeto. Esta es la base del microscopio simple o lupa.
\end{cajaconclusion}

\newpage

\subsection{Cuestión 1 - OPCIÓN B}
\label{subsec:3B_2000_jun_ord}

\begin{cajaenunciado}
Un rayo de luz monocromática que se propaga en el aire incide sobre la superficie del agua, cuyo índice de refracción respecto al aire es 1,33. Calcular el ángulo de incidencia para que el rayo reflejado sea perpendicular al rayo refractado.
\end{cajaenunciado}
\hrule

\subsubsection*{1. Tratamiento de datos y lectura}
\begin{itemize}
    \item \textbf{Medio de incidencia:} Aire, cuyo índice de refracción es $n_1 \approx 1$.
    \item \textbf{Medio de refracción:} Agua, cuyo índice de refracción es $n_2 = 1,33$.
    \item \textbf{Condición especial:} El ángulo entre el rayo reflejado y el rayo refractado es de $90^\circ$.
    \item \textbf{Incógnita:} Ángulo de incidencia ($\theta_i$).
\end{itemize}

\subsubsection*{2. Representación Gráfica}
\begin{figure}[H]
    \centering
    \fbox{\parbox{0.7\textwidth}{\centering \textbf{Reflexión y Refracción (Ángulo de Brewster)} \vspace{0.5cm} \textit{Prompt para la imagen:} "Diagrama de la superficie de separación horizontal entre dos medios, 'Aire ($n_1=1$)' arriba y 'Agua ($n_2=1,33$)' abajo. Dibuja la línea normal, perpendicular a la superficie. Un rayo de luz incide desde el aire con un ángulo de incidencia $\theta_i$. Dibuja el rayo reflejado hacia arriba con un ángulo de reflexión $\theta_r$. Dibuja el rayo refractado hacia abajo, desviándose hacia la normal, con un ángulo de refracción $\theta_t$. El ángulo entre el rayo reflejado y el refractado debe ser de $90^\circ$. Etiqueta todos los ángulos respecto a la normal." \vspace{0.5cm} % \includegraphics[width=0.6\linewidth]{angulo_brewster.png}
    }}
    \caption{Esquema del fenómeno con la condición de perpendicularidad.}
\end{figure}

\subsubsection*{3. Leyes y Fundamentos Físicos}
Se aplican las leyes fundamentales de la óptica geométrica:
\begin{itemize}
    \item \textbf{Ley de la Reflexión:} El ángulo de incidencia es igual al ángulo de reflexión ($\theta_i = \theta_r$).
    \item \textbf{Ley de la Refracción (Ley de Snell):} La relación entre los ángulos de incidencia y refracción y los índices de refracción es $n_1 \sin(\theta_i) = n_2 \sin(\theta_t)$.
    \item \textbf{Condición geométrica del problema:} Los ángulos alrededor del punto de incidencia sobre la línea normal suman $180^\circ$. En este caso, el ángulo del rayo reflejado, el ángulo de $90^\circ$ y el ángulo del rayo refractado, medidos desde la superficie, están sobre una línea. Con respecto a la normal, la relación es $\theta_r + 90^\circ + \theta_t = 180^\circ$.
\end{itemize}
El ángulo de incidencia que cumple esta condición se conoce como \textbf{ángulo de Brewster}.

\subsubsection*{4. Tratamiento Simbólico de las Ecuaciones}
A partir de la condición geométrica:
\begin{gather}
    \theta_r + \theta_t = 180^\circ - 90^\circ = 90^\circ
\end{gather}
Usando la ley de la reflexión, $\theta_r = \theta_i$, podemos sustituir:
\begin{gather}
    \theta_i + \theta_t = 90^\circ \implies \theta_t = 90^\circ - \theta_i
\end{gather}
Ahora, sustituimos esta expresión para $\theta_t$ en la Ley de Snell:
\begin{gather}
    n_1 \sin(\theta_i) = n_2 \sin(90^\circ - \theta_i)
\end{gather}
Utilizando la identidad trigonométrica $\sin(90^\circ - \alpha) = \cos(\alpha)$:
\begin{gather}
    n_1 \sin(\theta_i) = n_2 \cos(\theta_i)
\end{gather}
Reorganizamos la ecuación para despejar el ángulo de incidencia:
\begin{gather}
    \frac{\sin(\theta_i)}{\cos(\theta_i)} = \tan(\theta_i) = \frac{n_2}{n_1}
\end{gather}
Finalmente, el ángulo de incidencia es:
\begin{gather}
    \theta_i = \arctan\left(\frac{n_2}{n_1}\right)
\end{gather}

\subsubsection*{5. Sustitución Numérica y Resultado}
Sustituimos los valores de los índices de refracción:
\begin{gather}
    \theta_i = \arctan\left(\frac{1,33}{1}\right) = \arctan(1,33) \approx 53,06^\circ
\end{gather}
\begin{cajaresultado}
    El ángulo de incidencia debe ser $\boldsymbol{\theta_i \approx 53,06^\circ}$.
\end{cajaresultado}

\subsubsection*{6. Conclusión}
\begin{cajaconclusion}
Aplicando las leyes de la reflexión y la refracción junto con la condición geométrica de que los rayos reflejado y refractado son perpendiculares, se deduce la fórmula del ángulo de Brewster. Para la interfaz aire-agua, este ángulo es de aproximadamente $53,06^\circ$. A este ángulo de incidencia, la luz reflejada estaría completamente polarizada.
\end{cajaconclusion}

\newpage

% ======================================================================
\section{Bloque IV: Campo Eléctrico y Magnético}
\label{sec:em_2000_jun_ord}
% ======================================================================

\subsection{Problema 1 - OPCIÓN A}
\label{subsec:4A_2000_jun_ord}

\begin{cajaenunciado}
Un dipolo eléctrico está formado por dos cargas puntuales de +2µC y -2µC, distantes entre sí 6 cm. Calcular el campo y el potencial eléctrico:
\begin{enumerate}
    \item En un punto de la mediatriz del segmento que las une, distante 5 cm de cada carga.
    \item En un punto situado en la prolongación del segmento que las une y a 2 cm de la carga positiva.
\end{enumerate}
\textbf{Dato:} $K=9\times10^{9}$ SI.
\end{cajaenunciado}
\hrule

\subsubsection*{1. Tratamiento de datos y lectura}
\begin{itemize}
    \item \textbf{Carga positiva ($q_1$):} $q_1 = +2 \, \mu\text{C} = +2 \cdot 10^{-6} \, \text{C}$.
    \item \textbf{Carga negativa ($q_2$):} $q_2 = -2 \, \mu\text{C} = -2 \cdot 10^{-6} \, \text{C}$.
    \item \textbf{Distancia entre cargas ($2a$):} $2a = 6 \, \text{cm} = 0,06 \, \text{m}$. La semidistancia es $a = 0,03 \, \text{m}$.
    \item \textbf{Constante de Coulomb ($K$):} $K = 9 \cdot 10^9 \, \text{N}\cdot\text{m}^2/\text{C}^2$.
    \item \textbf{Apartado 1 (Punto P):} Situado en la mediatriz. Distancia a cada carga $r_1=r_2=5\,\text{cm}=0,05\,\text{m}$.
    \item \textbf{Apartado 2 (Punto Q):} Situado en el eje del dipolo. Distancia a la carga positiva $r_{1Q}=2\,\text{cm}=0,02\,\text{m}$.
    \item \textbf{Incógnitas:} Campo eléctrico ($\vec{E}$) y potencial ($V$) en los puntos P y Q.
\end{itemize}
Se establece un sistema de coordenadas donde $q_1$ está en $(-0,03, 0)$ y $q_2$ en $(+0,03, 0)$.

\subsubsection*{2. Representación Gráfica}
\begin{figure}[H]
    \centering
    \fbox{\parbox{0.45\textwidth}{\centering \textbf{Apartado 1: Punto en la Mediatriz} \vspace{0.5cm} \textit{Prompt para la imagen:} "Sistema de coordenadas XY. Una carga $+q$ en $(-a,0)$ y una carga $-q$ en $(+a,0)$. Un punto P en $(0,y)$. Dibuja el vector campo eléctrico $\vec{E}_1$ creado por $+q$ en P, saliendo de la carga. Dibuja el vector $\vec{E}_2$ creado por $-q$ en P, apuntando hacia la carga. Dibuja el vector resultante $\vec{E}_P$, que apunta horizontalmente hacia la derecha. Muestra que las componentes verticales de $\vec{E}_1$ y $\vec{E}_2$ se anulan." \vspace{0.5cm} % \includegraphics[width=0.9\linewidth]{dipolo_mediatriz.png}
    }}
    \hfill
    \fbox{\parbox{0.45\textwidth}{\centering \textbf{Apartado 2: Punto en el Eje} \vspace{0.5cm} \textit{Prompt para la imagen:} "Sistema de coordenadas XY. Una carga $+q$ en $(-a,0)$ y una carga $-q$ en $(+a,0)$. Un punto Q en el eje X, a la derecha de $-q$, en $(x_Q,0)$ con $x_Q > a$. Dibuja el vector $\vec{E}_1$ (creado por $+q$) en Q, apuntando a la derecha. Dibuja el vector $\vec{E}_2$ (creado por $-q$) en Q, apuntando a la izquierda. El vector $\vec{E}_2$ debe ser más largo que $\vec{E}_1$." \vspace{0.5cm} % \includegraphics[width=0.9\linewidth]{dipolo_eje.png}
    }}
    \caption{Esquemas para el cálculo del campo eléctrico del dipolo.}
\end{figure}

\subsubsection*{3. Leyes y Fundamentos Físicos}
Se aplica el \textbf{Principio de Superposición}. El campo y el potencial totales en un punto son la suma vectorial del campo y la suma escalar del potencial creados por cada carga individualmente.
\begin{itemize}
    \item \textbf{Campo eléctrico de una carga puntual:} $\vec{E} = K \frac{q}{r^2} \vec{u}_r$.
    \item \textbf{Potencial eléctrico de una carga puntual:} $V = K \frac{q}{r}$.
\end{itemize}

\subsubsection*{4. Tratamiento Simbólico de las Ecuaciones}
\paragraph{Apartado 1: Punto P en la mediatriz}
La altura del punto P sobre el eje es $y_P = \sqrt{r_1^2 - a^2}$. Por simetría, las componentes verticales del campo se anulan. El campo total es la suma de las componentes horizontales: $E_P = E_{1x} + E_{2x}$. Ambas apuntan hacia la derecha. El módulo de cada campo es $E_1 = E_2 = K \frac{|q|}{r_1^2}$. El campo total es $E_P = 2 E_1 \cos(\alpha)$, donde $\cos(\alpha) = a/r_1$.
\begin{gather}
    \vec{E}_P = 2 \left(K \frac{|q|}{r_1^2}\right) \frac{a}{r_1} \vec{i} = \frac{2K|q|a}{r_1^3} \vec{i}
\end{gather}
El potencial es la suma escalar:
\begin{gather}
    V_P = V_1 + V_2 = K\frac{q_1}{r_1} + K\frac{q_2}{r_2} = K\frac{q_1}{r_1} + K\frac{-q_1}{r_1} = 0
\end{gather}

\paragraph{Apartado 2: Punto Q en el eje}
El punto Q se encuentra a $r_{1Q}$ de $q_2$ (la carga negativa) y $r_{2Q} = 2a+r_{1Q}$ de $q_1$ (la carga positiva), pero el enunciado dice "a 2cm de la carga positiva", lo cual es ambiguo. Asumiendo la configuración $q_1(-a,0)$ y $q_2(+a,0)$, el punto Q está en $(+a+0.02, 0) = (0.05, 0)$.
Las distancias son: $r_{2Q}=0.02\,\text{m}$ a $q_2$ y $r_{1Q}=0.06+0.02=0.08\,\text{m}$ a $q_1$.
El campo es la suma vectorial (ambos campos en el eje x):
\begin{gather}
    \vec{E}_Q = \left( K \frac{q_1}{r_{1Q}^2} + K \frac{q_2}{r_{2Q}^2} \right) \vec{i} = K|q| \left( \frac{1}{r_{1Q}^2} - \frac{1}{r_{2Q}^2} \right) \vec{i}
\end{gather}
El potencial es la suma escalar:
\begin{gather}
    V_Q = K\frac{q_1}{r_{1Q}} + K\frac{q_2}{r_{2Q}} = K|q| \left( \frac{1}{r_{1Q}} - \frac{1}{r_{2Q}} \right)
\end{gather}

\subsubsection*{5. Sustitución Numérica y Resultado}
\paragraph{Apartado 1: Punto P}
\begin{gather}
    \vec{E}_P = \frac{2 \cdot (9 \cdot 10^9) \cdot (2 \cdot 10^{-6}) \cdot (0,03)}{(0,05)^3} \vec{i} = 8,64 \cdot 10^6 \vec{i} \, \text{N/C}
\end{gather}
\begin{cajaresultado}
    En el punto P: $\boldsymbol{\vec{E}_P = 8,64 \cdot 10^6 \vec{i} \, \textbf{N/C}}$ y $\boldsymbol{V_P = 0 \, \textbf{V}}$.
\end{cajaresultado}

\paragraph{Apartado 2: Punto Q}
Usando las distancias corregidas: $q_1(+2\mu\text{C})$ en $(-0.03,0)$ y $q_2(-2\mu\text{C})$ en $(+0.03,0)$. El punto Q está "a 2 cm de la carga positiva", en la prolongación, es decir, en $(-0.05, 0)$.
Distancias: $r_{1Q} = 0.02\,\text{m}$ a $q_1$, $r_{2Q} = 0.08\,\text{m}$ a $q_2$.
\begin{gather}
    \vec{E}_Q = K \left( \frac{q_1}{r_{1Q}^2} \vec{u}_{1Q} + \frac{q_2}{r_{2Q}^2} \vec{u}_{2Q} \right) = 9\cdot10^9 \left( \frac{2\cdot10^{-6}}{0.02^2}(-\vec{i}) + \frac{-2\cdot10^{-6}}{0.08^2}(-\vec{i}) \right) \nonumber \\
    \vec{E}_Q = 9\cdot10^9 \left( -5 \cdot 10^{-3} + 3.125 \cdot 10^{-4} \right)\vec{i} = -4.21875 \cdot 10^7 \vec{i} \, \text{N/C}
\end{gather}
\begin{gather}
    V_Q = 9\cdot10^9 \left( \frac{2\cdot10^{-6}}{0.02} + \frac{-2\cdot10^{-6}}{0.08} \right) = 9\cdot10^9 (10^{-4} - 2.5\cdot10^{-5}) = 6.75 \cdot 10^5 \, \text{V}
\end{gather}
\begin{cajaresultado}
    En el punto Q: $\boldsymbol{\vec{E}_Q \approx -4,22 \cdot 10^7 \vec{i} \, \textbf{N/C}}$ y $\boldsymbol{V_Q = 6,75 \cdot 10^5 \, \textbf{V}}$.
\end{cajaresultado}

\subsubsection*{6. Conclusión}
\begin{cajaconclusion}
En la mediatriz de un dipolo, el potencial eléctrico es siempre nulo y el campo eléctrico es perpendicular al eje del dipolo. En el eje del dipolo, tanto el campo como el potencial son no nulos, y sus valores dependen fuertemente de la distancia a cada una de las cargas, resultando en un campo muy intenso en las proximidades de las mismas.
\end{cajaconclusion}

\newpage

\subsection{Problema 1 - OPCIÓN B}
\label{subsec:4B_2000_jun_ord}

\begin{cajaenunciado}
Un electrón entra con velocidad constante $\vec{v}=10\vec{j}\,\text{m/s}$ en una región del espacio en la que existe un campo eléctrico uniforme $\vec{E}=20\vec{k}\,\text{N/C}$ y un campo magnético uniforme $\vec{B}=B_{0}\vec{i}\,\text{T}$. Se pide:
\begin{enumerate}
    \item Dibujar las fuerzas que actúan sobre el electrón (dirección y sentido), en el instante en que entra en la región en que existen los campos eléctrico y magnético.
    \item Calcular el valor de $B_0$ para que el movimiento del electrón sea rectilíneo y uniforme.
\end{enumerate}
\textbf{Nota:} Despreciar el campo gravitatorio.
\end{cajaenunciado}
\hrule

\subsubsection*{1. Tratamiento de datos y lectura}
\begin{itemize}
    \item \textbf{Partícula:} Electrón, carga $q = -e = -1,6 \cdot 10^{-19} \, \text{C}$.
    \item \textbf{Velocidad inicial:} $\vec{v} = 10 \vec{j} \, \text{m/s}$.
    \item \textbf{Campo eléctrico:} $\vec{E} = 20 \vec{k} \, \text{N/C}$.
    \item \textbf{Campo magnético:} $\vec{B} = B_0 \vec{i} \, \text{T}$.
    \item \textbf{Condición:} El movimiento es rectilíneo y uniforme (MRU).
    \item \textbf{Incógnitas:} Dibujo de las fuerzas y valor de $B_0$.
\end{itemize}

\subsubsection*{2. Representación Gráfica}
El dibujo de las fuerzas es una parte explícita de la pregunta.
\begin{figure}[H]
    \centering
    \fbox{\parbox{0.7\textwidth}{\centering \textbf{Selector de velocidades} \vspace{0.5cm} \textit{Prompt para la imagen:} "Un sistema de ejes coordenados 3D (X, Y, Z). Un electrón se muestra en el origen moviéndose a lo largo del eje Y positivo (vector $\vec{v}$). Dibuja el vector del campo eléctrico $\vec{E}$ apuntando en la dirección Z positiva. Dibuja el vector del campo magnético $\vec{B}$ apuntando en la dirección X positiva. Dibuja el vector de la fuerza eléctrica $\vec{F}_e$ sobre el electrón, apuntando en la dirección Z negativa (opuesta a $\vec{E}$ porque la carga es negativa). Dibuja el vector de la fuerza magnética $\vec{F}_m$ sobre el electrón, apuntando en la dirección Z positiva (resultado de $q(\vec{v} \times \vec{B})$ con $q<0$). Los vectores $\vec{F}_e$ y $\vec{F}_m$ deben dibujarse con la misma longitud para indicar que se anulan." \vspace{0.5cm} % \includegraphics[width=0.7\linewidth]{selector_velocidad.png}
    }}
    \caption{Diagrama de fuerzas sobre el electrón en los campos cruzados.}
\end{figure}

\subsubsection*{3. Leyes y Fundamentos Físicos}
La fuerza total que actúa sobre una partícula cargada en una región con campos eléctrico y magnético es la \textbf{Fuerza de Lorentz}:
$$ \vec{F} = \vec{F}_e + \vec{F}_m = q\vec{E} + q(\vec{v} \times \vec{B}) $$
Para que el movimiento sea rectilíneo y uniforme (MRU), la primera ley de Newton exige que la fuerza neta sobre la partícula sea cero:
$$ \sum \vec{F} = \vec{0} $$

\subsubsection*{4. Tratamiento Simbólico de las Ecuaciones}
\paragraph{1. Dirección y sentido de las fuerzas}
\begin{itemize}
    \item \textbf{Fuerza Eléctrica ($\vec{F}_e$):} $\vec{F}_e = q\vec{E}$. Como $q$ es negativa y $\vec{E}$ va en la dirección $+\vec{k}$, la fuerza eléctrica irá en la dirección \textbf{$-\vec{k}$}.
    \item \textbf{Fuerza Magnética ($\vec{F}_m$):} $\vec{F}_m = q(\vec{v} \times \vec{B})$. Primero calculamos el producto vectorial:
    $$ \vec{v} \times \vec{B} = (10\vec{j}) \times (B_0\vec{i}) = 10 B_0 (\vec{j} \times \vec{i}) = 10 B_0 (-\vec{k}) = -10 B_0 \vec{k} $$
    Ahora multiplicamos por la carga $q$ (negativa):
    $$ \vec{F}_m = q(-10 B_0 \vec{k}) = (-e)(-10 B_0 \vec{k}) = 10 e B_0 \vec{k} $$
    La fuerza magnética irá en la dirección \textbf{$+\vec{k}$}.
\end{itemize}
Como se ve en el dibujo, las dos fuerzas son opuestas.

\paragraph{2. Condición para MRU}
Imponemos que la fuerza neta sea cero:
\begin{gather}
    \vec{F}_e + \vec{F}_m = \vec{0} \implies q\vec{E} = -q(\vec{v} \times \vec{B}) \implies \vec{E} = -(\vec{v} \times \vec{B})
\end{gather}
En módulos, las fuerzas deben ser iguales:
\begin{gather}
    |\vec{F}_e| = |\vec{F}_m| \implies |q|E = |q|vB\sin(\theta)
\end{gather}
En nuestro caso, el ángulo $\theta$ entre $\vec{v}$ (eje Y) y $\vec{B}$ (eje X) es $90^\circ$, y $\sin(90^\circ)=1$.
\begin{gather}
    E = v B_0 \implies B_0 = \frac{E}{v}
\end{gather}

\subsubsection*{5. Sustitución Numérica y Resultado}
Sustituimos los módulos de E y v para encontrar $B_0$:
\begin{gather}
    B_0 = \frac{20 \, \text{N/C}}{10 \, \text{m/s}} = 2 \, \text{T}
\end{gather}
\begin{cajaresultado}
    El valor del módulo del campo magnético debe ser $\boldsymbol{B_0 = 2 \, \textbf{T}}$. Y como nos decía el enunciado estará orientado en el sentido positivo del eje OX $\vec{B} = 2 \vec{i} \, \text{T}$
\end{cajaresultado}

\subsubsection*{6. Conclusión}
\begin{cajaconclusion}
Para que un electrón atraviese una región de campos cruzados sin desviarse, la fuerza eléctrica y la fuerza magnética deben anularse mutuamente. Esto ocurre cuando son de igual módulo y dirección opuesta, lo que se consigue con un valor específico del campo magnético de 2 T. Este dispositivo se conoce como selector de velocidades, ya que solo las partículas con una velocidad específica ($v=E/B$) pueden pasar sin ser desviadas.
\end{cajaconclusion}

\newpage

% ======================================================================
\section{Bloque V: Física Moderna}
\label{sec:moderna_2000_jun_ord}
% ======================================================================

\subsection{Cuestión 1 - OPCIÓN A}
\label{subsec:5A_2000_jun_ord}

\begin{cajaenunciado}
Un electrón tiene una energía en reposo de 0,51 MeV. Si el electrón se mueve con una velocidad de 0,8c, se pide determinar su masa relativista, su cantidad de movimiento y su energía total.
\textbf{Datos:} Carga del electrón, $e=1,6\times10^{-19}$ C; Velocidad de la luz, $c=3\times10^{8}\,\text{m/s}$.
\end{cajaenunciado}
\hrule

\subsubsection*{1. Tratamiento de datos y lectura}
\begin{itemize}
    \item \textbf{Energía en reposo ($E_0$):} $E_0 = 0,51 \, \text{MeV} = 0,51 \cdot 10^6 \, \text{eV}$.
    \item Conversión a Julios: $E_0 = 0,51 \cdot 10^6 \, \text{eV} \times \frac{1,6 \cdot 10^{-19} \, \text{J}}{1 \, \text{eV}} = 8,16 \cdot 10^{-14} \, \text{J}$.
    \item \textbf{Velocidad del electrón ($v$):} $v = 0,8c$.
    \item \textbf{Velocidad de la luz ($c$):} $c = 3 \cdot 10^8 \, \text{m/s}$.
    \item \textbf{Incógnitas:} Masa relativista ($m$), cantidad de movimiento ($p$) y energía total ($E$).
\end{itemize}

\subsubsection*{2. Representación Gráfica}
\begin{figure}[H]
    \centering
    \fbox{\parbox{0.7\textwidth}{\centering \textbf{Efectos Relativistas} \vspace{0.5cm} \textit{Prompt para la imagen:} "Un gráfico conceptual mostrando cómo la masa relativista y la energía total de una partícula aumentan con la velocidad. El eje horizontal es la velocidad (v/c) de 0 a 1. El eje vertical representa masa/energía. Dibuja una curva para la masa relativista $m = \gamma m_0$ que comienza en $m_0$ en v=0 y tiende a infinito a medida que v tiende a c. Marca un punto en la curva correspondiente a v=0.8c." \vspace{0.5cm} % \includegraphics[width=0.6\linewidth]{masa_relativista.png}
    }}
    \caption{Aumento de la masa con la velocidad.}
\end{figure}

\subsubsection*{3. Leyes y Fundamentos Físicos}
Se utilizan las ecuaciones de la Relatividad Especial de Einstein:
\begin{itemize}
    \item \textbf{Factor de Lorentz ($\gamma$):} Es un factor que aparece en todas las transformaciones relativistas. $\gamma = \frac{1}{\sqrt{1 - v^2/c^2}}$.
    \item \textbf{Masa relativista ($m$):} La masa de un objeto depende de su velocidad: $m = \gamma m_0$, donde $m_0$ es la masa en reposo.
    \item \textbf{Energía total ($E$):} La energía total de una partícula es $E = mc^2 = \gamma m_0 c^2$.
    \item \textbf{Energía en reposo ($E_0$):} Es la energía asociada a la masa en reposo: $E_0 = m_0 c^2$. Por tanto, $E = \gamma E_0$.
    \item \textbf{Cantidad de movimiento relativista ($p$):} $p = mv = \gamma m_0 v$.
\end{itemize}

\subsubsection*{4. Tratamiento Simbólico de las Ecuaciones}
Primero, se calcula el factor de Lorentz $\gamma$ para la velocidad dada.
\begin{gather}
    \gamma = \frac{1}{\sqrt{1 - (0,8c)^2/c^2}} = \frac{1}{\sqrt{1 - 0,64}} = \frac{1}{\sqrt{0,36}} = \frac{1}{0,6}
\end{gather}
Luego, se aplican las fórmulas para las incógnitas:
\begin{itemize}
    \item \textbf{Masa relativista:} Se puede obtener a partir de la masa en reposo, $m_0 = E_0/c^2$. Entonces $m = \gamma m_0 = \gamma \frac{E_0}{c^2}$.
    \item \textbf{Energía total:} $E = \gamma E_0$.
    \item \textbf{Cantidad de movimiento:} $p = m v = \left(\gamma \frac{E_0}{c^2}\right) v$.
\end{itemize}

\subsubsection*{5. Sustitución Numérica y Resultado}
Calculamos el valor de $\gamma$:
\begin{gather}
    \gamma = \frac{1}{0,6} = \frac{5}{3} \approx 1,67
\end{gather}
\paragraph{a) Masa relativista}
Primero calculamos la masa en reposo $m_0$ a partir de $E_0$:
\begin{gather}
    m_0 = \frac{E_0}{c^2} = \frac{8,16 \cdot 10^{-14} \, \text{J}}{(3 \cdot 10^8 \, \text{m/s})^2} = 9,07 \cdot 10^{-31} \, \text{kg}
\end{gather}
Ahora la masa relativista:
\begin{gather}
    m = \gamma m_0 = \frac{5}{3} \cdot (9,07 \cdot 10^{-31} \, \text{kg}) \approx 1,51 \cdot 10^{-30} \, \text{kg}
\end{gather}
\begin{cajaresultado}
    La masa relativista del electrón es $\boldsymbol{m \approx 1,51 \cdot 10^{-30} \, \textbf{kg}}$.
\end{cajaresultado}

\paragraph{b) Energía total}
\begin{gather}
    E = \gamma E_0 = \frac{5}{3} \cdot (0,51 \, \text{MeV}) = 0,85 \, \text{MeV}
\end{gather}
\begin{cajaresultado}
    La energía total del electrón es $\boldsymbol{E = 0,85 \, \textbf{MeV}}$.
\end{cajaresultado}

\paragraph{c) Cantidad de movimiento}
\begin{gather}
    p = m v = (1,51 \cdot 10^{-30} \, \text{kg}) \cdot (0,8 \cdot 3 \cdot 10^8 \, \text{m/s}) = 3,624 \cdot 10^{-22} \, \text{kg}\cdot\text{m/s}
\end{gather}
\begin{cajaresultado}
    La cantidad de movimiento del electrón es $\boldsymbol{p \approx 3,62 \cdot 10^{-22} \, \textbf{kg}\cdot\textbf{m/s}}$.
\end{cajaresultado}

\subsubsection*{6. Conclusión}
\begin{cajaconclusion}
A velocidades cercanas a la de la luz, los efectos relativistas son significativos. A 0,8c, la masa del electrón aumenta en un 67\% respecto a su masa en reposo. Su energía total es también un 67\% mayor que su energía en reposo, y su cantidad de movimiento es considerablemente mayor que la que se calcularía con la fórmula clásica ($p=m_0 v$).
\end{cajaconclusion}

\newpage

\subsection{Cuestión 1 - OPCIÓN B}
\label{subsec:5B_2000_jun_ord}

\begin{cajaenunciado}
¿Con qué rapidez debe convertirse masa en energía para producir 20 MW?
\textbf{Dato:} Velocidad de la luz, $c=3\times10^{8}\,\text{m/s}$.
\end{cajaenunciado}
\hrule

\subsubsection*{1. Tratamiento de datos y lectura}
\begin{itemize}
    \item \textbf{Potencia producida ($P$):} $P = 20 \, \text{MW} = 20 \cdot 10^6 \, \text{W} = 20 \cdot 10^6 \, \text{J/s}$.
    \item \textbf{Velocidad de la luz ($c$):} $c = 3 \cdot 10^8 \, \text{m/s}$.
    \item \textbf{Incógnita:} La rapidez con la que se convierte la masa, es decir, la tasa de conversión de masa ($\frac{\Delta m}{\Delta t}$).
\end{itemize}

\subsubsection*{2. Representación Gráfica}
\begin{figure}[H]
    \centering
    \fbox{\parbox{0.7\textwidth}{\centering \textbf{Conversión Masa-Energía} \vspace{0.5cm} \textit{Prompt para la imagen:} "Una ilustración esquemática de un reactor nuclear o un sol estilizado. Mostrar una pequeña cantidad de 'masa' ($m$) entrando en el reactor/sol, y una gran cantidad de 'energía' ($E$) en forma de rayos de luz y calor saliendo. Incluir la ecuación de Einstein, $E=mc^2$, de forma prominente en el diagrama." \vspace{0.5cm} % \includegraphics[width=0.6\linewidth]{masa_energia.png}
    }}
    \caption{Concepto de conversión de masa en energía.}
\end{figure}

\subsubsection*{3. Leyes y Fundamentos Físicos}
El principio fundamental que rige este problema es la \textbf{equivalencia masa-energía} de Einstein, una de las conclusiones más importantes de la Relatividad Especial.
\begin{itemize}
    \item \textbf{Ecuación de Einstein:} $E = m c^2$. Esta fórmula establece que la masa puede ser convertida en energía y viceversa.
    \item \textbf{Potencia ($P$):} La potencia es la energía generada o consumida por unidad de tiempo. $P = \frac{\Delta E}{\Delta t}$.
\end{itemize}

\subsubsection*{4. Tratamiento Simbólico de las Ecuaciones}
Se relaciona la potencia con la conversión de masa. Si la energía producida proviene de la conversión de una cantidad de masa $\Delta m$, entonces $\Delta E = (\Delta m) c^2$.
Sustituyendo esto en la definición de potencia:
\begin{gather}
    P = \frac{(\Delta m) c^2}{\Delta t} = \left(\frac{\Delta m}{\Delta t}\right) c^2
\end{gather}
La incógnita es la tasa de conversión de masa, $\frac{\Delta m}{\Delta t}$. Despejándola de la ecuación anterior:
\begin{gather}
    \frac{\Delta m}{\Delta t} = \frac{P}{c^2}
\end{gather}
Esta ecuación nos dará la cantidad de masa que debe ser convertida en energía cada segundo para sostener la potencia de salida dada.

\subsubsection*{5. Sustitución Numérica y Resultado}
Sustituimos los valores numéricos proporcionados:
\begin{gather}
    \frac{\Delta m}{\Delta t} = \frac{20 \cdot 10^6 \, \text{J/s}}{(3 \cdot 10^8 \, \text{m/s})^2} = \frac{20 \cdot 10^6}{9 \cdot 10^{16}} \, \text{kg/s} \approx 2,22 \cdot 10^{-10} \, \text{kg/s}
\end{gather}
\begin{cajaresultado}
    La masa debe convertirse a una rapidez de $\boldsymbol{\approx 2,22 \cdot 10^{-10} \, \textbf{kg/s}}$.
\end{cajaresultado}

\subsubsection*{6. Conclusión}
\begin{cajaconclusion}
Para generar una potencia de 20 megavatios, es necesario convertir aproximadamente $2,22 \times 10^{-10}$ kilogramos de masa en energía cada segundo. Este resultado ilustra la inmensa cantidad de energía que se puede liberar a partir de una cantidad muy pequeña de masa, un principio fundamental en la energía nuclear y la astrofísica.
\end{cajaconclusion}

\newpage

% ======================================================================
\section{Bloque VI: Física Cuántica}
\label{sec:cuantica_2000_jun_ord}
% ======================================================================

\subsection{Cuestión 1 - OPCIÓN A}
\label{subsec:6A_2000_jun_ord}

\begin{cajaenunciado}
Describir el efecto fotoeléctrico y enumerar alguna de sus aplicaciones.
\end{cajaenunciado}
\hrule

\subsubsection*{1. Tratamiento de datos y lectura}
Es una cuestión teórica. Se debe explicar el fenómeno del efecto fotoeléctrico y listar sus usos prácticos.

\subsubsection*{2. Representación Gráfica}
\begin{figure}[H]
    \centering
    \fbox{\parbox{0.8\textwidth}{\centering \textbf{Esquema del Efecto Fotoeléctrico} \vspace{0.5cm} \textit{Prompt para la imagen:} "Un diagrama que ilustra el efecto fotoeléctrico. Muestra una superficie metálica (cátodo) dentro de un tubo de vacío. Rayos de luz (fotones, representados como paquetes de ondas con energía $E=h\nu$) inciden sobre la superficie. Como resultado, electrones (fotoelectrones) son eyectados de la superficie. Estos electrones son atraídos hacia otra placa (ánodo), creando una corriente eléctrica (foto-corriente) que es medida por un amperímetro en el circuito externo. Etiquetar la energía del fotón, la función de trabajo del metal ($\Phi$) y la energía cinética del electrón ($E_c$)." \vspace{0.5cm} % \includegraphics[width=0.7\linewidth]{efecto_fotoelectrico.png}
    }}
    \caption{Ilustración del fenómeno fotoeléctrico.}
\end{figure}

\subsubsection*{3. Leyes y Fundamentos Físicos}
\paragraph{Descripción del fenómeno}
El \textbf{efecto fotoeléctrico} consiste en la emisión de electrones por un material, generalmente metálico, cuando sobre él incide radiación electromagnética (como luz visible o ultravioleta).

\paragraph{Explicación de Einstein}
La física clásica no podía explicar las características de este efecto. Fue Albert Einstein, en 1905, quien proporcionó una explicación satisfactoria basándose en la idea de que la luz está cuantizada en paquetes de energía llamados \textbf{fotones}. La energía de cada fotón es proporcional a la frecuencia de la luz: $E_{fotón} = h\nu$, donde $h$ es la constante de Planck.

Las características clave y su explicación cuántica son:
\begin{itemize}
    \item \textbf{Frecuencia umbral ($\nu_0$):} Para cada material, existe una frecuencia mínima de la luz por debajo de la cual no se emiten electrones, sin importar cuán intensa sea la luz. Esto se debe a que un fotón individual debe tener suficiente energía para arrancar un electrón. La energía mínima necesaria se llama \textbf{función de trabajo} o trabajo de extracción ($\Phi$). Por lo tanto, se requiere que $h\nu \ge \Phi$.
    \item \textbf{Energía cinética de los electrones:} Si la energía del fotón es mayor que la función de trabajo, el exceso de energía se convierte en energía cinética del electrón emitido. La ecuación de Einstein para el efecto fotoeléctrico es:
    $$ E_{c, máx} = h\nu - \Phi $$
    La energía cinética de los fotoelectrones depende linealmente de la frecuencia de la luz, no de su intensidad.
    \item \textbf{Instantaneidad:} La emisión de electrones es prácticamente instantánea, incluso con luces de muy baja intensidad. Esto se explica porque la energía se transfiere en "paquetes" discretos (fotones), y la interacción fotón-electrón es un evento único e instantáneo.
    \item \textbf{Intensidad de la luz:} Un aumento en la intensidad de la luz (a una frecuencia constante por encima del umbral) corresponde a un mayor número de fotones incidiendo por segundo. Esto resulta en un mayor número de electrones emitidos, es decir, una mayor corriente eléctrica, pero no aumenta la energía cinética individual de los electrones.
\end{itemize}

\subsubsection*{5. Aplicaciones del Efecto Fotoeléctrico}
El efecto fotoeléctrico es la base de numerosos dispositivos tecnológicos:
\begin{itemize}
    \item \textbf{Células fotovoltaicas (paneles solares):} Convierten la energía de la luz solar directamente en energía eléctrica.
    \item \textbf{Sensores de imagen (CCD y CMOS):} Utilizados en cámaras digitales, teléfonos móviles y telescopios. Cada píxel del sensor es un dispositivo fotoeléctrico que convierte la luz que recibe en una señal eléctrica.
    \item \textbf{Fotodiodos y fototransistores:} Se utilizan en una gran variedad de sensores, como en los controles remotos, lectores de códigos de barras y sistemas de comunicación por fibra óptica.
    \item \textbf{Sistemas de apertura automática de puertas:} Un haz de luz (a menudo infrarroja) incide sobre un fotodetector. Cuando una persona interrumpe el haz, la corriente se corta y el mecanismo de la puerta se activa.
    \item \textbf{Fotomultiplicadores:} Dispositivos extremadamente sensibles capaces de detectar niveles muy bajos de luz, incluso fotones individuales.
\end{itemize}

\subsubsection*{6. Conclusión}
\begin{cajaconclusion}
El efecto fotoeléctrico es un fenómeno cuántico fundamental que demuestra la naturaleza corpuscular de la luz. Su explicación por Einstein fue un pilar en el desarrollo de la mecánica cuántica. Más allá de su importancia teórica, es el principio operativo detrás de tecnologías clave en la generación de energía, la imagen digital y la sensórica moderna.
\end{cajaconclusion}

\newpage

\subsection{Cuestión 1 - OPCIÓN B}
\label{subsec:6B_2000_jun_ord}

\begin{cajaenunciado}
¿Por qué el espectro del hidrógeno tiene muchas líneas si el átomo de hidrógeno tiene un solo electrón?
\end{cajaenunciado}
\hrule

\subsubsection*{1. Tratamiento de datos y lectura}
Es una pregunta teórica que requiere explicar el origen de las múltiples líneas espectrales del átomo más simple.

\subsubsection*{2. Representación Gráfica}
\begin{figure}[H]
    \centering
    \fbox{\parbox{0.8\textwidth}{\centering \textbf{Niveles de Energía y Transiciones en el Hidrógeno} \vspace{0.5cm} \textit{Prompt para la imagen:} "Un diagrama de niveles de energía para el átomo de hidrógeno. El eje vertical representa la energía. Dibuja líneas horizontales para los niveles de energía cuantizados, etiquetados como n=1 (estado fundamental), n=2, n=3, n=4, etc., acercándose entre sí a medida que n aumenta. Dibuja varias flechas verticales apuntando hacia abajo, representando transiciones electrónicas entre diferentes niveles. Agrupa las transiciones en series: (1) Flechas que terminan en n=1 (Serie de Lyman, en el ultravioleta). (2) Flechas que terminan en n=2 (Serie de Balmer, en el visible). (3) Flechas que terminan en n=3 (Serie de Paschen, en el infrarrojo). Cada flecha debe tener una etiqueta como 'emisión de fotón'." \vspace{0.5cm} % \includegraphics[width=0.8\linewidth]{niveles_hidrogeno.png}
    }}
    \caption{El origen de las series espectrales en las transiciones electrónicas.}
\end{figure}

\subsubsection*{3. Leyes y Fundamentos Físicos}
La respuesta se basa en el \textbf{modelo atómico de Bohr} y los principios de la mecánica cuántica, que establecen que la energía del electrón en un átomo está \textbf{cuantizada}.

\begin{itemize}
    \item \textbf{Postulados de Bohr:}
    \begin{enumerate}
        \item El electrón no puede orbitar al núcleo a cualquier distancia, sino solo en un número limitado de órbitas estables, llamadas \textbf{niveles de energía} o estados estacionarios. Cada nivel tiene una energía específica y bien definida.
        \item Mientras el electrón permanece en uno de estos niveles, no irradia energía.
        \item El electrón puede "saltar" de un nivel de energía a otro. Para pasar de un nivel de menor energía ($E_i$) a uno de mayor energía ($E_f$), debe absorber un fotón con una energía exactamente igual a la diferencia de energía entre los niveles: $\Delta E = E_f - E_i = h\nu$.
        \item Cuando el electrón decae de un nivel de mayor energía ($E_i$) a uno de menor energía ($E_f$), emite un fotón cuya energía es, de nuevo, igual a la diferencia de energía: $\Delta E = E_i - E_f = h\nu$.
    \end{enumerate}
\end{itemize}

\paragraph{Origen de las múltiples líneas}
Aunque un átomo de hidrógeno individual solo tiene un electrón, una muestra de gas de hidrógeno contiene miles de millones de átomos. Cuando se excita esta muestra (por ejemplo, mediante una descarga eléctrica), los electrones de los diferentes átomos son llevados a diversos niveles de energía excitados (n=2, n=3, n=4, etc.).

Estos electrones excitados no permanecen en esos niveles altos de forma indefinida, sino que tienden a volver a niveles más bajos y, finalmente, al estado fundamental (n=1). El punto clave es que \textbf{existen múltiples caminos para volver al estado fundamental}.

Por ejemplo, un electrón en el nivel n=4 puede:
\begin{itemize}
    \item Caer directamente a n=1.
    \item Caer a n=2 y luego a n=1.
    \item Caer a n=3 y luego a n=1.
    \item Caer a n=3, luego a n=2 y finalmente a n=1.
\end{itemize}
Cada una de estas caídas (transiciones) implica la emisión de un fotón con una energía específica y, por lo tanto, una frecuencia y longitud de onda específicas. Como en una muestra de gas ocurren todas las transiciones posibles simultáneamente en diferentes átomos, se observa un espectro compuesto por muchas líneas discretas. Cada línea corresponde a una transición electrónica única entre dos niveles de energía permitidos.

\subsubsection*{6. Conclusión}
\begin{cajaconclusion}
La existencia de múltiples líneas en el espectro del hidrógeno, a pesar de tener un solo electrón, no se debe al número de electrones, sino al \textbf{número de niveles de energía cuantizados y permitidos} para ese electrón. La gran variedad de posibles transiciones o "saltos" entre estos niveles de energía da lugar a la emisión de fotones de muchas energías diferentes, lo que se traduce en un espectro de líneas rico y característico.
\end{cajaconclusion}